%!TEX TS-options = --shell-escape
%!TEX TS-program = pdflatex
\documentclass[%
   10pt,              % Schriftgroesse
                 % wird an andere Pakete weitergereicht
   a4paper,           % Seitengroesse
   DIV10,             % Textbereichsgroesse (siehe Koma Skript Dokumentation !)
]{scrartcl}%     Klassen: scrartcl, scrreprt, scrbook, article
% -------------------------------------------------------------------------

\usepackage[utf8]{inputenc} % Font Encoding, benoetigt fuer Umlaute
%\usepackage[ngerman]{babel}   % Spracheinstellung


\usepackage{ulem}
\usepackage{graphicx}
\usepackage{amsfonts}
\usepackage{amsmath}
\usepackage{hyperref}
\usepackage{enumitem}
\usepackage{tikz}
\usepackage{multirow}
\usepackage{listings}
\usepackage{ifthen}
\usepackage{todonotes}
\usepackage{mathtools}
\usetikzlibrary{automata,arrows}


% Definition des Headers
\usepackage{geometry}
\geometry{a4paper, top=3cm, left=3cm, right=3cm, bottom=3cm, headsep=0mm, footskip=0mm}
\renewcommand{\baselinestretch}{1.3}\normalsize
\renewcommand{\labelenumi}{\alph{enumi})}
\newcommand{\norm}[1]{\left\lVert#1\right\rVert}
\def\header#1#2#3#4#5#6#7{\pagestyle{empty}
\noindent
\begin{minipage}[t]{0.6\textwidth}
\begin{flushleft}
\textbf{#4}\\% Fach
#6\\% Semester
#2  % Tutor 
\end{flushleft}
\end{minipage}
\begin{minipage}[t]{0.4\textwidth}
\begin{flushright}
\points{#7}% Punktetabelle
\vspace*{0.2cm}
#5%  Names
\end{flushright}
\end{minipage}

\begin{center}
{\Large\textbf{ Assignment #1}} % Blatt

{(Abgabe am #3)} % Abgabedatum
\end{center}
}

\newenvironment{vartab}[1]
{
    \begin{tabular}{ |c@{} *{#1}{c|} } %\hline
}{
    \end{tabular}
}

\newcommand{\myformat}[1]{& #1}

\newcommand{\entry}[1]{
  \edef\result{\csvloop[\myformat]{#1}}
  \result \\ \hline
}

\newcommand{\numbers}[1]{
  \newcounter{ctra}
\setcounter{ctra}{1}
\whiledo {\value{ctra} < #1}%
{%
  \myformat{\thectra}
  \stepcounter{ctra}%
}
\myformat{\thectra}
}
\newcommand{\emptyLine}[1]{
  \newcounter{ctra1}
\setcounter{ctra}{1}
\whiledo {\value{ctra1} < #1}%
{%
  \myformat{\hspace*{0.5cm}}
  \stepcounter{ctra1}%
}
}


\newcommand{\points}[1]{
\newcounter{colmns}
\setcounter{colmns}{#1}
\stepcounter{colmns}
  \begin{vartab}{\thecolmns}
    \numbers{#1} & $\sum$\\\hline
    \emptyLine{\thecolmns}\\
  \end{vartab}

}

\begin{document}
%\header{Blatt}{Tutor}{Abgabedatum}{Vorlesung}{Bearbeiter}{Semester}{Anzahl Aufgaben}
\header{6}{}{05. Juni 2018}{ML: Algo \& Theory}{Florence Lopez (3878792), florence.lopez@student.uni-tuebingen.de \\ Jennifer Them (3837649), jennifer.them@student.uni-tuebingen.de}{SS 18}{3}

\section*{Exercise 2}
\begin{itemize}
	\item[d.)] A support vector is a vector, whose Lagrangian multipliers are non-zero. In the case of a hard-margin SVM the support vectors lie on the margin. In the case of a soft-margin SVM the support vectors can lie either on the margin, inside the margin or on the wrong side of the hyperplane. 
\end{itemize}

\section*{Exercise 3}
\begin{itemize}
	\item[a.)] The primal problem is: minimize$_{x\in \mathbb{R}} f(x) = x^4 - 10x^2 + x$ subject to $g(x) = x^2 - 2x - 3 \leq 0$.\newline
	\noindent The feasible set of this problem can be computed by trying out. One can see that for $x \in [-1, 3], g(x)$ stays $\leq 0$, since for $x = -2, g(x) = 5$ and for $x = 4, g(x) = 5$, too, which is both $> 0$.\newline
	\noindent To solve the primal, one has to compute: $\frac{\delta f(x)}{\delta x} = 4x^3 - 20x + 1 \stackrel{!}{=} 0$. We solved this using WolframAlpha and got the following results: 
	\item[b.)] The Langrangian to this problem is:\newline
	\noindent $L(x, \alpha) = f(x) + \alpha g(x) = x^4 - 10x^2 + x + \alpha x^2 - 2\alpha x - 3\alpha$. The derivative with respect to $x$ is: $\frac{\delta L(x, \alpha)}{\delta x} = x^3 - 20x + 1 +2\alpha - 2\alpha = x^3 - 20x + 1$.\newline
	\noindent The dual function can be computed by: $h(\alpha) = \text{inf}_x L(x, \alpha) =  \text{inf}_x x^4 - 10x^2 + x + \alpha x^2 - 2\alpha x - 3\alpha$. We know from the exercise that the dual function $h(\alpha)$ is optimal for $\alpha = 0.5$. Therefore we can calculate the optimal solution of the dual function:\newline 
	\noindent $h(\alpha) = h(0.5) = \text{inf}_x  x^4 - 10x^2 + x + \frac{1}{2} x^2 - x - \frac{3}{2} = x^4 - \frac{19}{2}x^2 - \frac{3}{2}$. To get the value of the infimum, we have to set the derivative of $h(0.5)$ to zero:\newline
	\noindent $\frac{\delta h(0.5)}{\delta x} = 4x^3 - 19x \stackrel{!}{=} 0 \Leftrightarrow 4x^3 = 19x \Leftrightarrow x^2 = \frac{19}{4} \Leftrightarrow x = $.  
\end{itemize}

\end{document}
%!TEX TS-options = --shell-escape
%!TEX TS-program = pdflatex
\documentclass[%
   10pt,              % Schriftgroesse
                 % wird an andere Pakete weitergereicht
   a4paper,           % Seitengroesse
   DIV10,             % Textbereichsgroesse (siehe Koma Skript Dokumentation !)
]{scrartcl}%     Klassen: scrartcl, scrreprt, scrbook, article
% -------------------------------------------------------------------------

\usepackage[utf8]{inputenc} % Font Encoding, benoetigt fuer Umlaute
%\usepackage[ngerman]{babel}   % Spracheinstellung


\usepackage{ulem}
\usepackage{graphicx}
\usepackage{amsfonts}
\usepackage{amsmath}
\usepackage{hyperref}
\usepackage{enumitem}
\usepackage{tikz}
\usepackage{multirow}
\usepackage{listings}
\usepackage{ifthen}
\usepackage{todonotes}
\usepackage{mathtools}
\usetikzlibrary{automata,arrows}


% Definition des Headers
\usepackage{geometry}
\geometry{a4paper, top=3cm, left=3cm, right=3cm, bottom=3cm, headsep=0mm, footskip=0mm}
\renewcommand{\baselinestretch}{1.3}\normalsize
\renewcommand{\labelenumi}{\alph{enumi})}
\newcommand{\norm}[1]{\left\lVert#1\right\rVert}
\def\header#1#2#3#4#5#6#7{\pagestyle{empty}
\noindent
\begin{minipage}[t]{0.6\textwidth}
\begin{flushleft}
\textbf{#4}\\% Fach
#6\\% Semester
#2  % Tutor 
\end{flushleft}
\end{minipage}
\begin{minipage}[t]{0.4\textwidth}
\begin{flushright}
\points{#7}% Punktetabelle
\vspace*{0.2cm}
#5%  Names
\end{flushright}
\end{minipage}

\begin{center}
{\Large\textbf{ Assignment #1}} % Blatt

{(Abgabe am #3)} % Abgabedatum
\end{center}
}

\newenvironment{vartab}[1]
{
    \begin{tabular}{ |c@{} *{#1}{c|} } %\hline
}{
    \end{tabular}
}

\newcommand{\myformat}[1]{& #1}

\newcommand{\entry}[1]{
  \edef\result{\csvloop[\myformat]{#1}}
  \result \\ \hline
}

\newcommand{\numbers}[1]{
  \newcounter{ctra}
\setcounter{ctra}{1}
\whiledo {\value{ctra} < #1}%
{%
  \myformat{\thectra}
  \stepcounter{ctra}%
}
\myformat{\thectra}
}
\newcommand{\emptyLine}[1]{
  \newcounter{ctra1}
\setcounter{ctra}{1}
\whiledo {\value{ctra1} < #1}%
{%
  \myformat{\hspace*{0.5cm}}
  \stepcounter{ctra1}%
}
}


\newcommand{\points}[1]{
\newcounter{colmns}
\setcounter{colmns}{#1}
\stepcounter{colmns}
  \begin{vartab}{\thecolmns}
    \numbers{#1} & $\sum$\\\hline
    \emptyLine{\thecolmns}\\
  \end{vartab}

}

\begin{document}
%\header{Blatt}{Tutor}{Abgabedatum}{Vorlesung}{Bearbeiter}{Semester}{Anzahl Aufgaben}
\header{10}{}{03. Juli 2018}{ML: Algo \& Theory}{Florence Lopez (3878792), florence.lopez@student.uni-tuebingen.de \\ Jennifer Them (3837649), jennifer.them@student.uni-tuebingen.de}{SS 18}{3}

\section*{Exercise 1}
\begin{itemize}
	\item[a.)] We have to prove that $\lambda_n \leq 2$. From the Rayleigh principle we know that $\lambda_n = \max_v v^T L v$, where $v \in \mathbb{R}^n$ is normalized, meaning that is has length = 1. Because $L$ is a normalized Laplacian, we get the following equations:\\
	\noindent $v^T L v = \frac{1}{2} \sum_{i, j = 1}^{n} a_{ij} (\frac{v_i}{\sqrt{d_i}} - \frac{v_j}{\sqrt{d_j}})^2$\\
	\noindent $ = \frac{1}{2} \sum_{i, j = 1}^{n} a_{ij} ((\frac{v_i}{\sqrt{d_i}})^2 - 2\frac{v_i}{\sqrt{d_i}}\frac{v_j}{\sqrt{d_j}} + (\frac{v_j}{\sqrt{d_j}})^2)$ $ = \frac{1}{2} \sum_{i, j = 1}^{n} a_{ij} (\frac{v_i^2}{d_i} - 2\frac{v_i v_j}{\sqrt{d_i d_j}} + \frac{v_j^2}{d_j})$\\
	\noindent $ = \frac{1}{2} \sum_{i, j = 1}^{n} a_{ij} \frac{v_i^2}{d_i} - \sum_{i,j = 1}^{n} a_{ij} \frac{v_i v_j}{\sqrt{d_i d_j}} +  \frac{1}{2} \sum_{i,j=1}^{n} a_{ij} \frac{v_j^2}{d_j}$ $ = \sum_{i = 1}^{n} (\sum_{j = 1}^{n} a_{ij}) \frac{v_i^2}{d_i} - \sum_{i,j = 1}^{n} a_{ij} \frac{v_i v_j}{\sqrt{d_i d_j}}$\\
	$ = \sum_{i = 1}^{n} v_i^2 - \sum_{i,j = 1}^{n} a_{ij} \frac{v_i v_j}{\sqrt{d_i d_j}}$ $ = 1 - \sum_{i,j = 1}^{n} \frac{a_{ij} }{\sqrt{d_i d_j}} v_i v_j$.\\
	\noindent Since $v$ is normalized, the following to relations hold: $-1 \leq v \leq 1$ and $-1 \leq v_iv_j \leq 1$. We can now assume that $\sum_{i,j = 1}^{n} \frac{a_{ij} }{\sqrt{d_i d_j}} \leq 1$, since the sum of the adjacency entries is divided by other sums of the same value. By multiplying $\sum_{i,j = 1}^{n} \frac{a_{ij} }{\sqrt{d_i d_j}}$ with some number $v_i v_j$ between -1 and 1, we would still get a number between -1 and 1, leading to $\sum_{i,j = 1}^{n} \frac{a_{ij} }{\sqrt{d_i d_j}} v_i v_j \geq -1$. With all those relations, we can now conclude that:\\ 
	\noindent$\lambda_n = \max_v v^T L v = \max_v 1 - \sum_{i,j = 1}^{n} \frac{a_{ij} }{\sqrt{d_i d_j}} \leq 1 - (-1) = 2$. 
	\item[b.)] If $G$ is a complete graph on $n$ vertices, we have $A = 1 - I$, which is a matrix that has 1 in every entry, except on the diagonal. Then $D = diag(n-1, \dots, n-1)$, which leads to $D - A$ being $(n-1)$ on the diagonal and -1 everywhere else. If we multiply $D^{-\frac{1}{2}}$ on the left and on the right, we divide each entry by $(n-1)$, which leads to $L$ being $1$ on the diagonal and $\frac{-1}{n-1}$ everywhere else.\\
	\noindent If we set the eigenvector $v_1$ to $(1, \dots, 1)$, we get $L v_1 = 0 v_1$, which means that we have an eigenvalue of 0. If we set the eigenvector to one of the other $n-1$ possible ones, we get $v_i = (0, \dots, 0,1,-1,0, \dots, 0)$. The 1 here stands at position $i-1$ and the -1 stands at position $i$. Then we would have $L v_i = \frac{n}{n-1} v_i$, because $L$ would contain 0 at every place, except for $i-1$ and $i$, which leads to $1 \cdot 1 - 1 \cdot \frac{-1}{n-1} = \frac{n}{n-1}$ and $1 \cdot \frac{-1}{n-1} - 1 \cdot 1 = \frac{-n}{n-1} = (-1) \cdot  \frac{n}{n-1}$.
	\item[c.)] If $G$ is a complete bipartite graph on $n+m$ vertices, than $A$ consists of a $n \times n$ block of zeros on the top left, of a $m \times m$ block of zeros on the bottom right and two $n \times m$ and $m\times n$ blocks of non-zeros on the remaining places. Then $D$ has a $m$ on the diagonal for the first $n$ vertices and a $n$ for the remaining $m$ vertices. Therefore $D-A$ has $m$ or $n$ written on its diagonal and -1 in the upper right and lower left blocks and 0 on all other positions. If we now multiply $D^{-\frac{1}{2}}$ on the left and on the right, we would now divide each entry by either $m, n$ or $\sqrt{mn}$, so $L$ would be 1 on the diagonal and $\frac{-1}{\sqrt{nm}}$ on the upper right blocks and lower left blocks and 0 everywhere else.\\
	\noindent If we set the eigenvector to $v_1$ consisting of $n$ times $\sqrt{m}$ and $m$ times $\sqrt{n}$, we would get $L v_1 = 0 v_1$, since each entry is $\sqrt{m} \cdot 1 + m \cdot \sqrt{n} \cdot \frac{-1}{\sqrt{mn}} = \sqrt{m} - \frac{\sqrt{n} \sqrt{m}^2}{\sqrt{nm}} = \sqrt{m} - \sqrt{m} = 0$ and the analogous for $n$ and $m$ changed. This means we have an eigenvalue of 0.\\
	\noindent If we set the eigenvector to $v_n$ consisting of $n$ times $-\sqrt{m}$ and $m$ times $\sqrt{n}$, we would get $L v_n = 2 v_n$, since each entry is $-\sqrt{m} \cdot 1 + m \cdot \sqrt{n} \cdot \frac{-1}{\sqrt{mn}} = -\sqrt{m} - \frac{\sqrt{n} \sqrt{m}^2}{\sqrt{nm}} = 2(-\sqrt{m})$ and the analogous for $n$ and $m$ changed. This means we have an eigenvalue of 2.\\
	\noindent For the remaining $n-1 + m-1$ eigenvectors, we set $v_i = (0, \cdots, 0, 1, -1, 0, \cdots, 0)$ with a 1 at position $i-1$ and -1 at position $i$. We would then get $L v_i = 1 v_i$, since $L v_i$ contains 0 at every position, except for position $i-1$ and $i$, where we would have values 1 and -1. This means we would have $(n+m-2)$ eigenvalues with value 1. 
	\item[d.)] -
\end{itemize}
\section*{Exercise 2}
see code. 
\section*{Exercise 3}
\end{document}